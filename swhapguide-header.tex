  %
% put gray background around verbatim text
%
% Reference
% <https://tex.stackexchange.com/questions/411194/package-soul-error-reconstruction-failed>

\usepackage{xcolor}
\usepackage{soul}

\definecolor{inlineBG}{HTML}{F3F3F3}  % same as GitHub Flavored Markdown
\sethlcolor{inlineBG}

\let\OldTexttt\texttt
\renewcommand{\texttt}[1]{\sethlcolor{inlineBG}{\ttfamily\hl{\mbox{#1}}}}

%
% configure lstlisting package
%
\usepackage{lstconfig}

%
% from: https://tex.stackexchange.com/questions/30845/how-to-redefine-lstinline-to-automatically-highlight-or-draw-frames-around-all
%

\usepackage{etoolbox}
\usepackage{atbegshi,ifthen,listings,tikz}

% change this to customize the appearance of the highlight
\tikzstyle{highlighter} = [
  inlineBG,
  line width = \baselineskip,
]

% enable these two lines for a more human-looking highlight
%\usetikzlibrary{decorations.pathmorphing}
%\tikzstyle{highlighter} += [decorate, decoration = random steps]

% implementation of the core highlighting logic; do not change!
\newcounter{highlight}[page]
\newcommand{\tikzhighlightanchor}[1]{\ensuremath{\vcenter{\hbox{\tikz[remember picture, overlay]{\coordinate (#1 highlight \arabic{highlight});}}}}}
\newcommand{\bh}[0]{\stepcounter{highlight}\tikzhighlightanchor{begin}}
\newcommand{\eh}[0]{\tikzhighlightanchor{end}}
\AtBeginShipout{\AtBeginShipoutUpperLeft{\ifthenelse{\value{highlight} > 0}{\tikz[remember picture, overlay]{\foreach \stroke in {1,...,\arabic{highlight}} \draw[highlighter] (begin highlight \stroke) -- (end highlight \stroke);}}{}}}
%--------------------------


\makeatletter %   Redefine macros from listings package:
\newtoggle{@InInlineListing}%
\togglefalse{@InInlineListing}%

\renewcommand\lstinline[1][]{%
    \leavevmode\bgroup\toggletrue{@InInlineListing}\bh % \hbox\bgroup --> \bgroup
      \def\lst@boxpos{b}%
      \lsthk@PreSet\lstset{flexiblecolumns,#1}%
      \lsthk@TextStyle
      \@ifnextchar\bgroup{\afterassignment\lst@InlineG \let\@let@token}%
                         \lstinline@}%

\def\lst@LeaveAllModes{%
    \ifnum\lst@mode=\lst@nomode
        \expandafter\lsthk@EndGroup\iftoggle{@InInlineListing}{\eh{}}{}%
    \else
        \expandafter\egroup\expandafter\lst@LeaveAllModes
    \fi%
    }
\makeatother



  \usepackage{swhap}
  \usepackage{lastpage}
  \shortreporttitle{SWHAP}
  \reportkind{SWHAP Guidelines}
  \reporttitle{Software Heritage Acquisition Process}
  \reportversion{1.0-1}
  \reportauthorlist{{\bf Authors:} & Laura Bussi, Dept. of Computer Science, University of Pisa $\langle${\tt l.bussi1@studenti.unipi.it}$\rangle$\\&Roberto Di Cosmo, Software Heritage, Inria and University of Paris $\langle${\tt roberto@dicosmo.org}$\rangle$\\& Carlo Montangero, Dept. of Computer Science, University of Pisa $\langle${\tt carlo@montangero.eu}$\rangle$\\& Guido Scatena, Dept. of Computer Science, University of Pisa $\langle${\tt guido.scatena@unipi.it}$\rangle$\\}
  \setcounter{secnumdepth}{2}
  \reportabstract{
    The source code of landmark legacy software is particularly important: it sheds
  insights in the history of the evolution of a technology that has changed the
  world, and tells a story of the humans that dedicated their lives to it.\\
  Rescuing it is urgent, collecting and curating it is a complex task that
  requires significant human intervention.\\
  This document presents the first version of SWHAP, the Software Heritage
  Acquisition Process: a protocol for the collection and preservation of software
  of historical and scientific relevance. SWHAP results from a fruitful
  collaboration of the University of Pisa with Software Heritage in this area of
  research, under the auspices of UNESCO, and has been validated on a selection of
  software source code produced in the Pisa area over the past 50 years.\\[2em]
  \paragraph{Acknowledments}
  L. Bussi wants to acknowledge the Software Heritage Foundation for the
  scholarship that supported her work and the Department of Computer
  Science of the University of Pisa for hosting her while working on
  SWHAPPE.
  \vfill
  \paragraph{License}
  This work is distributed under the terms of the \href{https://creativecommons.org/licenses/by/4.0/}{Creative Commons license CC-BY 4.0}
  }
