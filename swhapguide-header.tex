  %
% put gray background around verbatim text
%
% Reference
% <https://tex.stackexchange.com/questions/411194/package-soul-error-reconstruction-failed>

%\PassOptionsToPackage{table}{xcolor}
%\PassOptionsToPackage{dvipscolor}{xcolor}
\usepackage[table,dvipscolor,usenames]{xcolor}
\usepackage{soul}

\definecolor{inlineBG}{HTML}{F3F3F3}  % same as GitHub Flavored Markdown
\sethlcolor{inlineBG}

\let\OldTexttt\texttt
\renewcommand{\texttt}[1]{\sethlcolor{inlineBG}{\ttfamily\hl{\mbox{#1}}}}

%
% configure lstlisting package
%
  \lstset{ 
  backgroundcolor=\color{gray!10},   % choose the background color; you must add \usepackage{color} or \usepackage{xcolor}; should come as last argument
  basicstyle=\normalsize,        % the size of the fonts that are used for the code
  breakatwhitespace=false,         % sets if automatic breaks should only happen at whitespace
  breaklines=true,                 % sets automatic line breaking
  captionpos=b,                    % sets the caption-position to bottom
  commentstyle=\color{mygreen},    % comment style
  deletekeywords={source},         % if you want to delete keywords from the given language
  escapeinside={\%*}{*)},          % if you want to add LaTeX within your code
  emph={git,curl},
  emphstyle={\color{blue}},
  extendedchars=true,              % lets you use non-ASCII characters; for 8-bits encodings only, does not work with UTF-8
  firstnumber=1000,                % start line enumeration with line 1000
  frame=shadowbox,	                   % adds a frame around the code
  frameround=tttt,
  rulesepcolor=\color{gray},
  columns=flexible,
  keepspaces=true,                 % keeps spaces in text, useful for keeping indentation of code (possibly needs columns=flexible)
  keywordstyle=\color{blue},       % keyword style
  language=bash,                   % the language of the code
  morekeywords={git,curl}          % if you want to add more keywords to the set
  numbers=left,                    % where to put the line-numbers; possible values are (none, left, right)
  numbersep=5pt,                   % how far the line-numbers are from the code
  numberstyle=\tiny\color{gray}, % the style that is used for the line-numbers
  rulecolor=\color{black},         % if not set, the frame-color may be changed on line-breaks within not-black text (e.g. comments (green here))
  showspaces=false,                % show spaces everywhere adding particular underscores; it overrides 'showstringspaces'
  showstringspaces=false,          % underline spaces within strings only
  showtabs=false,                  % show tabs within strings adding particular underscores
  stepnumber=2,                    % the step between two line-numbers. If it's 1, each line will be numbered
  stringstyle=\color{cyan},     % string literal style
  tabsize=2,	                   % sets default tabsize to 2 spaces
}

%
% from: https://tex.stackexchange.com/questions/30845/how-to-redefine-lstinline-to-automatically-highlight-or-draw-frames-around-all
%

\usepackage{etoolbox}
\usepackage{atbegshi,ifthen,listings,tikz}

% change this to customize the appearance of the highlight
\tikzstyle{highlighter} = [
  inlineBG,
  line width = \baselineskip,
]

% enable these two lines for a more human-looking highlight
%\usetikzlibrary{decorations.pathmorphing}
%\tikzstyle{highlighter} += [decorate, decoration = random steps]

% implementation of the core highlighting logic; do not change!
\newcounter{highlight}[page]
\newcommand{\tikzhighlightanchor}[1]{\ensuremath{\vcenter{\hbox{\tikz[remember picture, overlay]{\coordinate (#1 highlight \arabic{highlight});}}}}}
\newcommand{\bh}[0]{\stepcounter{highlight}\tikzhighlightanchor{begin}}
\newcommand{\eh}[0]{\tikzhighlightanchor{end}}
\AtBeginShipout{\AtBeginShipoutUpperLeft{\ifthenelse{\value{highlight} > 0}{\tikz[remember picture, overlay]{\foreach \stroke in {1,...,\arabic{highlight}} \draw[highlighter] (begin highlight \stroke) -- (end highlight \stroke);}}{}}}
%--------------------------


\makeatletter %   Redefine macros from listings package:
\newtoggle{@InInlineListing}%
\togglefalse{@InInlineListing}%

\renewcommand\lstinline[1][]{%
    \leavevmode\bgroup\toggletrue{@InInlineListing}\bh % \hbox\bgroup --> \bgroup
      \def\lst@boxpos{b}%
      \lsthk@PreSet\lstset{flexiblecolumns,#1}%
      \lsthk@TextStyle
      \@ifnextchar\bgroup{\afterassignment\lst@InlineG \let\@let@token}%
                         \lstinline@}%

\def\lst@LeaveAllModes{%
    \ifnum\lst@mode=\lst@nomode
        \expandafter\lsthk@EndGroup\iftoggle{@InInlineListing}{\eh{}}{}%
    \else
        \expandafter\egroup\expandafter\lst@LeaveAllModes
    \fi%
    }
\makeatother

%
% Create nice section headers using tikz and titlesec
%

\usepackage[explicit]{titlesec}

\titleformat{\section}
  {\gdef\sectionlabel{}
   \normalfont\sffamily\Large\bfseries\scshape}
  {\gdef\sectionlabel{\thesection)\ }}{0pt}
  {   \begin{tikzpicture}%[remember picture, overlay]
        \node[rectangle,
              rounded corners=14pt,inner sep=8pt,
              fill=red!90]
              {\color{white}~~\sectionlabel#1~~};
       \end{tikzpicture}
  }
\titlespacing*{\section}{-20pt}{30pt}{30pt}

\titleformat{\subsection}
  {\gdef\subsectionlabel{}
   \normalfont\sffamily\large\bfseries\scshape}
  {\gdef\subsectionlabel{\thesubsection)\ }}{0pt}
  {   \begin{tikzpicture}%[remember picture, overlay]
        \node[rectangle,
              rounded corners=10pt,inner sep=4pt,
              fill=red!90]
              {\color{white}~~\subsectionlabel#1~~};
       \end{tikzpicture}
  }
\titlespacing*{\subsection}{-10pt}{10pt}{10pt}

\titleformat{\subsubsection}
  {\gdef\subsubsectionlabel{}
   \normalfont\sffamily\bfseries\scshape}
  {\gdef\subsubsectionlabel{\thesubsubsection)\ }}{0pt}
  {   \begin{tikzpicture}%[remember picture, overlay]
        \node[rectangle,
              rounded corners=8pt,inner sep=4pt,
              fill=red!90]
              {\color{white}~~\subsubsectionlabel#1~~};
       \end{tikzpicture}
  }
\titlespacing*{\subsubsection}{0pt}{10pt}{8pt}

  \usepackage{swhap}
  \usepackage{lastpage}
  \shortreporttitle{SWHAP}
  \reportkind{SWHAP Guidelines}
  \reporttitle{Software Heritage Acquisition Process}
  \reportversion{1.1}
  \reportauthorlist{{\bf Authors:} & Laura Bussi, Dept. of Computer Science, University of Pisa $\langle${\tt l.bussi1@studenti.unipi.it}$\rangle$\\&Roberto Di Cosmo, Software Heritage, Inria and University of Paris $\langle${\tt roberto@dicosmo.org}$\rangle$\\&Mathilde Fichen, Software Heritage and CNAM $\langle${\tt mathilde.fichen@inria.fr}$\rangle$\\& Carlo Montangero, Dept. of Computer Science, University of Pisa $\langle${\tt carlo@montangero.eu}$\rangle$\\& Guido Scatena, Dept. of Computer Science, University of Pisa $\langle${\tt guido.scatena@unipi.it}$\rangle$\\}
  \setcounter{secnumdepth}{2}
  \reportabstract{
    The source code of landmark legacy software is particularly important: it sheds
  insights in the history of the evolution of a technology that has changed the
  world, and tells a story of the humans that dedicated their lives to it.\\
  Rescuing it is urgent, collecting and curating it is a complex task that
  requires significant human intervention.\\
This document is a practical, simplified, step-by-step guide to the Software Heritage Acquisition Process (SWHAP). Its aim is to provide tools and guidelines to properly archive source code of historical and scientific relevance in the Software Heritage Universal Archive. This updated guide builds upon the original SWHAP guide, which was initially published in 2019 as a result of a fruitful collaboration between the University of Pisa and Software Heritage in this area of research, under the auspices of UNESCO. It has been validated using a selection of software source code produced in the Pisa area over the past 50 years.

The original guide remains the reference for the global approach adopted and the conceptual choices made. This updated guide aims to offer a simplified, practical implementation of SWHAP.
  \\[2em]
  \paragraph{Acknowledments}
  The authors would like to acknowledge UNESCO for providing the funding that supported the development of this guide.
  \vfill
  \paragraph{License}
  This work is distributed under the terms of the \href{https://creativecommons.org/licenses/by/4.0/}{Creative Commons license CC-BY 4.0}
  }
