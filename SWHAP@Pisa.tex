\PassOptionsToPackage{unicode=true}{hyperref} % options for packages loaded elsewhere
\PassOptionsToPackage{hyphens}{url}
%
\documentclass[]{article}
\usepackage{lmodern}
\usepackage{amssymb,amsmath}
\usepackage{ifxetex,ifluatex}
\usepackage{fixltx2e} % provides \textsubscript
\ifnum 0\ifxetex 1\fi\ifluatex 1\fi=0 % if pdftex
  \usepackage[T1]{fontenc}
  \usepackage[utf8]{inputenc}
  \usepackage{textcomp} % provides euro and other symbols
\else % if luatex or xelatex
  \usepackage{unicode-math}
  \defaultfontfeatures{Ligatures=TeX,Scale=MatchLowercase}
\fi
% use upquote if available, for straight quotes in verbatim environments
\IfFileExists{upquote.sty}{\usepackage{upquote}}{}
% use microtype if available
\IfFileExists{microtype.sty}{%
\usepackage[]{microtype}
\UseMicrotypeSet[protrusion]{basicmath} % disable protrusion for tt fonts
}{}
\IfFileExists{parskip.sty}{%
\usepackage{parskip}
}{% else
\setlength{\parindent}{0pt}
\setlength{\parskip}{6pt plus 2pt minus 1pt}
}
\usepackage{hyperref}
\hypersetup{
            pdftitle={The Software Heritage Acquisition Presentation (SWH-Stories)},
            pdfborder={0 0 0},
            breaklinks=true}
\urlstyle{same}  % don't use monospace font for urls
\usepackage{listings}
\newcommand{\passthrough}[1]{#1}
\setlength{\emergencystretch}{3em}  % prevent overfull lines
\providecommand{\tightlist}{%
  \setlength{\itemsep}{0pt}\setlength{\parskip}{0pt}}
\setcounter{secnumdepth}{0}

% set default figure placement to htbp
\makeatletter
\def\fps@figure{htbp}
\makeatother

  %
% put gray background around verbatim text
%
% Reference
% <https://tex.stackexchange.com/questions/411194/package-soul-error-reconstruction-failed>

%\PassOptionsToPackage{table}{xcolor}
%\PassOptionsToPackage{dvipscolor}{xcolor}
\usepackage[table,dvipscolor,usenames]{xcolor}
\usepackage{soul}

\definecolor{inlineBG}{HTML}{F3F3F3}  % same as GitHub Flavored Markdown
\sethlcolor{inlineBG}

\let\OldTexttt\texttt
\renewcommand{\texttt}[1]{\sethlcolor{inlineBG}{\ttfamily\hl{\mbox{#1}}}}

%
% configure lstlisting package
%
  \lstset{ 
  backgroundcolor=\color{gray!10},   % choose the background color; you must add \usepackage{color} or \usepackage{xcolor}; should come as last argument
  basicstyle=\normalsize,        % the size of the fonts that are used for the code
  breakatwhitespace=false,         % sets if automatic breaks should only happen at whitespace
  breaklines=true,                 % sets automatic line breaking
  captionpos=b,                    % sets the caption-position to bottom
  commentstyle=\color{mygreen},    % comment style
  deletekeywords={source},         % if you want to delete keywords from the given language
  escapeinside={\%*}{*)},          % if you want to add LaTeX within your code
  emph={git,curl},
  emphstyle={\color{blue}},
  extendedchars=true,              % lets you use non-ASCII characters; for 8-bits encodings only, does not work with UTF-8
  firstnumber=1000,                % start line enumeration with line 1000
  frame=shadowbox,	                   % adds a frame around the code
  frameround=tttt,
  rulesepcolor=\color{gray},
  columns=flexible,
  keepspaces=true,                 % keeps spaces in text, useful for keeping indentation of code (possibly needs columns=flexible)
  keywordstyle=\color{blue},       % keyword style
  language=bash,                   % the language of the code
  morekeywords={git,curl}          % if you want to add more keywords to the set
  numbers=left,                    % where to put the line-numbers; possible values are (none, left, right)
  numbersep=5pt,                   % how far the line-numbers are from the code
  numberstyle=\tiny\color{gray}, % the style that is used for the line-numbers
  rulecolor=\color{black},         % if not set, the frame-color may be changed on line-breaks within not-black text (e.g. comments (green here))
  showspaces=false,                % show spaces everywhere adding particular underscores; it overrides 'showstringspaces'
  showstringspaces=false,          % underline spaces within strings only
  showtabs=false,                  % show tabs within strings adding particular underscores
  stepnumber=2,                    % the step between two line-numbers. If it's 1, each line will be numbered
  stringstyle=\color{cyan},     % string literal style
  tabsize=2,	                   % sets default tabsize to 2 spaces
}

%
% from: https://tex.stackexchange.com/questions/30845/how-to-redefine-lstinline-to-automatically-highlight-or-draw-frames-around-all
%

\usepackage{etoolbox}
\usepackage{atbegshi,ifthen,listings,tikz}

% change this to customize the appearance of the highlight
\tikzstyle{highlighter} = [
  inlineBG,
  line width = \baselineskip,
]

% enable these two lines for a more human-looking highlight
%\usetikzlibrary{decorations.pathmorphing}
%\tikzstyle{highlighter} += [decorate, decoration = random steps]

% implementation of the core highlighting logic; do not change!
\newcounter{highlight}[page]
\newcommand{\tikzhighlightanchor}[1]{\ensuremath{\vcenter{\hbox{\tikz[remember picture, overlay]{\coordinate (#1 highlight \arabic{highlight});}}}}}
\newcommand{\bh}[0]{\stepcounter{highlight}\tikzhighlightanchor{begin}}
\newcommand{\eh}[0]{\tikzhighlightanchor{end}}
\AtBeginShipout{\AtBeginShipoutUpperLeft{\ifthenelse{\value{highlight} > 0}{\tikz[remember picture, overlay]{\foreach \stroke in {1,...,\arabic{highlight}} \draw[highlighter] (begin highlight \stroke) -- (end highlight \stroke);}}{}}}
%--------------------------


\makeatletter %   Redefine macros from listings package:
\newtoggle{@InInlineListing}%
\togglefalse{@InInlineListing}%

\renewcommand\lstinline[1][]{%
    \leavevmode\bgroup\toggletrue{@InInlineListing}\bh % \hbox\bgroup --> \bgroup
      \def\lst@boxpos{b}%
      \lsthk@PreSet\lstset{flexiblecolumns,#1}%
      \lsthk@TextStyle
      \@ifnextchar\bgroup{\afterassignment\lst@InlineG \let\@let@token}%
                         \lstinline@}%

\def\lst@LeaveAllModes{%
    \ifnum\lst@mode=\lst@nomode
        \expandafter\lsthk@EndGroup\iftoggle{@InInlineListing}{\eh{}}{}%
    \else
        \expandafter\egroup\expandafter\lst@LeaveAllModes
    \fi%
    }
\makeatother

%
% Create nice section headers using tikz and titlesec
%

\usepackage[explicit]{titlesec}

\titleformat{\section}
  {\gdef\sectionlabel{}
   \normalfont\sffamily\Large\bfseries\scshape}
  {\gdef\sectionlabel{\thesection)\ }}{0pt}
  {   \begin{tikzpicture}%[remember picture, overlay]
        \node[rectangle,
              rounded corners=14pt,inner sep=8pt,
              fill=red!90]
              {\color{white}~~\sectionlabel#1~~};
       \end{tikzpicture}
  }
\titlespacing*{\section}{-20pt}{30pt}{30pt}

\titleformat{\subsection}
  {\gdef\subsectionlabel{}
   \normalfont\sffamily\large\bfseries\scshape}
  {\gdef\subsectionlabel{\thesubsection)\ }}{0pt}
  {   \begin{tikzpicture}%[remember picture, overlay]
        \node[rectangle,
              rounded corners=10pt,inner sep=4pt,
              fill=red!90]
              {\color{white}~~\subsectionlabel#1~~};
       \end{tikzpicture}
  }
\titlespacing*{\subsection}{-10pt}{10pt}{10pt}

\titleformat{\subsubsection}
  {\gdef\subsubsectionlabel{}
   \normalfont\sffamily\bfseries\scshape}
  {\gdef\subsubsectionlabel{\thesubsubsection)\ }}{0pt}
  {   \begin{tikzpicture}%[remember picture, overlay]
        \node[rectangle,
              rounded corners=8pt,inner sep=4pt,
              fill=red!90]
              {\color{white}~~\subsubsectionlabel#1~~};
       \end{tikzpicture}
  }
\titlespacing*{\subsubsection}{0pt}{10pt}{8pt}

  \usepackage{swhap}
  \usepackage{lastpage}
  \shortreporttitle{SWHAP}
  \reportkind{SWHAP Guidelines}
  \reporttitle{Software Heritage Acquisition Process}
  \reportauthorlist{&Roberto Di Cosmo, Software Heritage, Inria and University of Paris $\langle${\tt roberto@dicosmo.org}$\rangle$\\& Carlo Montangero, Dept. of Computer Science, University of Pisa $\langle${\tt carlo@montangero.eu}$\rangle$\\& Guido Scatena, Dept. of Computer Science, University of Pisa $\langle${\tt guido.scatena@unipi.it}$\rangle$\\}
  \setcounter{secnumdepth}{2}
  \reportabstract{
    The presentation of acquired source code of landmark legacy software is particularly important.
This document presents the first version of the integration of WikiMedia and ScinceStories.io 
presentation into SWHAP, the Software Heritage Acquisition Process: a protocol for the collection and preservation of software
of historical and scientific relevance. SWHAP results from a fruitful
collaboration of the University of Pisa with Software Heritage in this area of
research, under the auspices of UNESCO, and has been validated on a selection of
software source code produced in the Pisa area over the past 50 years.\\[2em]
  \paragraph{Acknowledments}
  TBD
  \vfill
  \paragraph{License}
  This work is distributed under the terms of the \href{https://creativecommons.org/licenses/by/4.0/}{Creative Commons license CC-BY 4.0}
  }
\usepackage[]{biblatex}
\addbibresource{swhap.bib}

\title{The Software Heritage Acquisition Presentation (SWH-Stories)}
\author{true \and true \and true}
\date{30 November 2021}

\begin{document}
\maketitle

{
\setcounter{tocdepth}{3}
\tableofcontents
}
\hypertarget{introduction}{%
\section{Introduction}\label{introduction}}

\textbf{Create a SWH-story}

The suggested process to document the recovered source code with a story
in the SWH-stories website
\href{https://swh.stories.k2.services/stories/}{{https://swh.stories.k2.services/stories/}}
has two phases: - Collect, where the Presentation designer collects the
images, videos, documents to be published. For each item, he also
gathers in a suitably structered \emph{inventory} in the Workbench the
information nedeed to insure that once uploaded in the appropriate
Wikimedia data base the item is satisfactorily self documenting. The
Presentation designer should also take care that the item can be granted
a pubblic domain license, as required by Wikimedia policies. - Publish,
where the Web engineer, in this case the Wikimedia expert, uploads the
items in such a way that they are best presented exploiting the STORIES
SERVICE at {[}{http://stage.stories.k2.services/publisher/}{]}.

The story will be available at
{[}(https://swh.stories.k2.services/stories/)\{.underline\}{]}. Since
the story is dinamically constructed at each access, the above process
can be freely iterated, to add new elements or to correct the presented
information.

TBD : Choose between inserting the sentence \emph{The process is
described in detail in ???} or elaborate here the description above. The
presentation process, abstract view \{\#sec:processabs\}

TBD : Presentation designer might be abstract, and instantiated as
SWH-stories designer in this version.

\hypertarget{presenting-in-swh-stories}{%
\section{Presenting in SWH-stories}\label{presenting-in-swh-stories}}

This way of presenting the recovered source code is inspired by the
\href{https://sciencestories.io}{{https://sciencestories.io}} website.

TDB : decide between an \emph{embedded} or an \emph{external} approach.
In the embedded case we insert here a description as abstract as
possible of the process to create a SWH-story, and see how to put other
information in the rest of the document. In the external case, we refer
to a new document, based on Morane and Kat's report, completed with a
section related to the support in the adjourned template of SWHAPPE.

\hypertarget{resources-in-the-process}{%
\subsection{Resources in the process}\label{resources-in-the-process}}

\begin{itemize}
\tightlist
\item
  SWHAP Repository
\item
  Wikidata
\item
  Wikimedia Commons
\item
  Software Heritage
\item
  Publisher Workspace (?)
\end{itemize}

\hypertarget{swhap-repository-structure}{%
\subsection{SWHAP Repository
structure}\label{swhap-repository-structure}}

\hypertarget{cover-page}{%
\subsubsection{\texorpdfstring{\textbf{Cover
Page}}{Cover Page}}\label{cover-page}}

Cover page TODO:

\hypertarget{story-inventory}{%
\subsubsection{\texorpdfstring{\textbf{Story
Inventory}}{Story Inventory}}\label{story-inventory}}

The story invetory is the central worikng point of the presentation
process of SWHAP. It consists in a document and a set of folders. The
document serves ad catalogue of collected items and guides the process
of tranfer acquired matherials to wiki*.

\hypertarget{presentation-moments}{%
\subsubsection{\texorpdfstring{\textbf{Presentation
Moments}}{Presentation Moments}}\label{presentation-moments}}

The story of a software, as presented by ScienceStories.io, is divided
in ``moments''.

\hypertarget{people-moments}{%
\subsubsection{People Moments}\label{people-moments}}

as a software story is also a story about people \ldots{} This section,
and the realtive folders, provides informatins and storage for the
images, videos, documents, etc. related to the people involved in the
software project.

The people story moments fall in two classes, with a dedicated
sub-folder each:

\begin{itemize}
\tightlist
\item
  \emph{media\_gallery}, for images and videos, and
\item
  \emph{library}, for documents.
\end{itemize}

An entry should be added to the
\href{https://github.com/Unipisa/Softi-Workbench/blob/structure_review/additional-materials/swh_stories_workplace/StoryInventory.md}{StoryInventory}
file for each added item, following the pattern offered in the dedicated
section.

\hypertarget{software-moments}{%
\subsubsection{Software Moments}\label{software-moments}}

This section, and the relative folders, provides storage for the images,
videos, documents, etc. related to the software.

The software story moments fall in four classes, with a dedicated
sub-folder each:

\begin{itemize}
\tightlist
\item
  \textbf{code\_listing}, for PDF versions of the source code, annotated
  with permalinks to the SWH archive,
\item
  \textbf{media\_gallery}, for images of the original source code
  recovered from non digital documents,
\item
  \textbf{library} for links to documents related to the software, and
\item
  \textbf{videos} for related videos.
\end{itemize}

An entry should be added to the
\href{https://github.com/Unipisa/Softi-Workbench/blob/structure_review/additional-materials/swh_stories_workplace/StoryInventory.md}{StoryInventory}
file for each added item, following the pattern offered in the dedicated
section.

\hypertarget{wikidata-entities}{%
\paragraph{Wikidata Entities}\label{wikidata-entities}}

in this section we annotate the entities to be created

\hypertarget{roles-in-the-process}{%
\subsection{Roles in the process}\label{roles-in-the-process}}

\hypertarget{collector}{%
\subsubsection{\texorpdfstring{\textbf{Collector}}{Collector}}\label{collector}}

The curator is in charge of retirve alle the mandatory information abot
the collected matherials, as, for instance, the authors, the dates and
the copytright.

\hypertarget{curator}{%
\subsubsection{\texorpdfstring{\textbf{Curator}}{Curator}}\label{curator}}

The curator is in charge of recustructing the story behind the software,
about people and places. Moreover, the curator is in charge of selecting
and creating a presentable objects for acquired matherials. For instance
he select and crop the acquired images, he create the annotaded pdf
version of relevant pieces of code listng, with links to the actual code
archived in Software Heritage by the SWHID.

\hypertarget{presentation-designer-and-web-engineer}{%
\subsubsection{\texorpdfstring{\textbf{Presentation designer and Web
engineer}*}{Presentation designer and Web engineer*}}\label{presentation-designer-and-web-engineer}}

Though most of the presentations of the archived software will be on
line, the abilities to design the contents of a presentation should be
considered separately from the technical ones. For instance, in the case
of the SWH-stories, the presentation designer should be competent in the
topic addressed by the code, to be able to search and select the items
to be inserted in the story. On the other side, the web engineer should
be acknowledgeable of Wikimedia and the other tools involved in creating
the stories.

\hypertarget{visitor}{%
\subsubsection{\texorpdfstring{\textbf{Visitor}}{Visitor}}\label{visitor}}

The visitor can navigate to https://swh.stories.k2.services/ where he
can find the collection of software acquired and presented by Software
Heritage.

\hypertarget{sec:walkthrough}{%
\section{A walkthrough on a running example}\label{sec:walkthrough}}

\printbibliography[title=Bibliography]

\end{document}
